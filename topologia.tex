\documentclass[12pt]{article}
\usepackage[utf8]{inputenc}
\usepackage[spanish]{babel}
\usepackage{amsmath}
\usepackage{amsfonts}
\usepackage{amssymb}
\usepackage{amsthm}
\usepackage{blindtext}
\usepackage{mathtools}
\usepackage{graphicx}
\usepackage{latexsym}
\usepackage{cancel}
\usepackage[left=2cm,top=2cm,right=2cm,bottom=2cm]{geometry}
\usepackage[all]{xy}
\usepackage{cancel}
\usepackage{pictexwd}
\usepackage{parskip}
\usepackage{pgfplots}
\pgfplotsset{compat=1.15}
\usepackage{mathrsfs}
\usepackage{vmargin}


\DeclarePairedDelimiter\Floor\lfloor\rfloor
\DeclarePairedDelimiter\Ceil\lceil\rceil


\newtheorem{theorem}{Teorema}[section]
\newtheorem{definicion}[theorem]{Definición}
\newtheorem{proposition}[theorem]{Proposición}
\newtheorem{lemma}{Lema}[theorem]
\newtheorem{definition}[theorem]{Definición}
\newtheorem{example}{Ejemplo}[theorem]
\newtheorem{corolario}{Corolario}[theorem]
\newtheorem{observation}{Observación}[theorem]
\newtheorem{properties}{Propiedades}[theorem]
\providecommand{\abs}[1]{\lvert#1\rvert}
\providecommand{\norm}[1]{\lVert#1\rVert}


\author{Pablo Pallàs}
\title{Topología general}
\setlength{\parindent}{10pt}


\begin{document}
\rmfamily
\maketitle
\tableofcontents
\parindent= 0cm

\section{Espacios topológicos}
\subsection{Espacios topológicos}
Sea $X$ un conjunto y $\mathcal{P}(X) = \lbrace A : A \subset X \rbrace$ el conjunto de sus partes, entonces:

\begin{definition}Una \textbf{topología} sobre un conjunto $X$ es un subconjunto $\tau \subset \mathcal{P}(X)$ que satisface: 
 \renewcommand{\theenumi}{\roman{enumi}} %Números arábigos
\begin{enumerate}
\item El conjunto vacío $\emptyset$ y el conjunto total $X$ pertenecen a $\tau$.
\item La unión arbitraria de elementos de $\tau$ también pertenece a $\tau$.
\item La intersección finita de elementos de $\tau$ también pertenece a $\tau$.
\end{enumerate}
El par $(X,\tau)$ lo denominaremos \textbf{espacio topológico} y a los elementos de $\tau$ los llamaremos \textbf{abiertos}.
\end{definition}

Es decir, podríamos decir que una topología es una colección de subconjuntos que contiene al vacío y al total, y que es cerrada para las uniones arbitrarias y las intersecciones finitas.

\begin{example}Sea $X$ un conjunto arbitrario y $\tau_D = \mathcal{P}(X)$. Entonces, $\tau_D$ es una topología en $X$ ya que contiene a todos los subconjuntos de $X$, en particular al vacío y al total, es cerrada para las uniones arbitrarias y para las intersecciones finitas. A esta topología la denominaremos \textbf{topología discreta}, y al conjunto $X$ dotada de esta topología \textbf{espacio discreto}.
\end{example}
\begin{example}Sea $X$ un conjunto arbitrario y $\tau_I = \lbrace \emptyset, X \rbrace$. Entonces la colección $\tau_I$ es una topología sobre $X$: contiene al vacío y al total, la unión de ambos es $X \in \tau_I$ y la intersección es $\emptyset \in \tau_I$. Esta topología la denominaremos \textbf{topología indiscreta}, y es la topología más simple que puede tener un conjunto. A un conjunto $X$ dotado con esta topología lo denominaremos \textbf{espacio indiscreto}.
\end{example}

\begin{definition}Dos topologías $\tau_1, \tau_2$ sobre un conjunto $X$ se dicen \textbf{comparables} si $\tau_1 \subset \tau_2$ ó $\tau_2 \subset \tau_1$. Si $\tau_1 \subset \tau_2$ diremos que $\tau_2$ es más \textbf{fina} (tiene más abiertos) que $\tau_1$.
\end{definition}

\section{Aplicaciones continuas y homeomorfismos}
\section{Separación y numerabilidad}
\section{Espacios métricos}
\section{Compacidad}
\section{Conexión}

\end{document}